\chapter{Evaluation of the Project/Lessons Learned}
\section{Project Management}
\subsection{Team Roles??}
\subsection{Time Management??}
\subsection{Attendance}
\subsection{Backlog and Task Management}
A lesson we clearly learned at the end of the project was to minimize where and how task management is done. 
To manage the product- and sprint backlogs we, initially, decided to use a spreadsheet in google docs and Trello. The spreadsheet were used to plan the sprints, break down tasks, do estimation and assignment, whereas, Trello was used to keep track of the progress of the tasks. However, the customer wanted us to use issues actively on gitHub. We then started using gitHub task management features in addition to our already existing methods for doing so. Instead of helping us, this simply led to confusing issues and team members not knowing where to go to find issues. Different team members used different channels to update their progress, and Trello was quickly abandoned by most.

The reason we did not use gitHub to manage our project completely, even after having it suggested by the customer, was that we did not want to add tasks relating to the report to our repository on gitHub. We believed it was in the customers interest to keep them separate. However, towards the end of the project, we realized that the customer thought we managed all our tasks and backlog through gitHub, and was surprised to learn that we had another system we used as well. We all agreed that this made it difficult for the product owner to keep track of what was going on. 

Had we known about all the task management features of gitHub at the start of the project, we see, in hindsight, that we should have simply managed our project through this one channel from the start. This insight into gitHub and how it also eased communications with the customer has been a great lessons from the project.
\section{Customer and Vague requirements??}
The team members were unfamiliar with Lean principles of development, and the biggest challenge initially was to understand this way of working. Expecting to quickly have established clear requirements to base our work on, it was difficult for the team to respond to what we, at the time, felt was an indecisive customer. We see in hindsight, that the customer's approach to the development process was one we were not familiar with. It took us several weeks to realize that we could and should be more proactive in coming up with solutions and that we were equally capable as the customer when it came to brainstorm solutions.

Had we realized this sooner, we could probably have had greater progress in a couple of sprints. Some of our placeholder items in the final solution might have been fully implemented, but this is in no way guaranteed as they were still dependent on factors outside our control. 

One example is the fitness indicator on our web site. Initially we believed we were going to collect data, do calculations and present this to the user. However, we were never able to settle with the customer which types of data should form the basis for any calculations.Later in the project, it became clear that Adapt, the research project we are a part of, where the ones making decisions about this data, but that they had not yet decided either what kind of data should be measured. Thus, even if we had been proactive is designing solutions for how to display the data, the functionality could not have been fully implemented as the product owner was not even made completely aware of the data to be collected and measured.

\section{Development Tools??}