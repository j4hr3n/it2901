\chapter{Lessons Learned}\label{lessons}
\section{Project Management}
\subsection{Team Roles}
At the start of the project we assigned roles to all team members. These roles proved to be very suggestive as the roles had a tendency of blur and be shifted throughout the course of the project.

From this we learned that is important to always make sure the assigned roles is up to date. If they blur or shift they should be updated according to that. 

\subsection{Time Management}
During this project it has from time to time been hard to manage time when working together. This comes as a result of the fact that we have not been clear enough about what is to be done at all times. This in turn caused some irritation as little-to-none work was done on the sessions where this was a problem.

A lesson learned from this is that one should always make sure that one agrees on what to work on.

\subsection{Attendance}
Throughout the course of the project we have attempted to host daily meetings. This proven to be quite hard as all team members have different time schedules and degrees of responsibility on other projects. The team members attended as much as possible, but at times the attendance at meetings were very low or nonexisting.

This lesson has taught us the importance of attending to meetings. Everyone should understand the importance of attending. In that way they may be compelled to attend a little more.

\subsection{Backlog and Task Management}
A lesson we clearly learned at the end of the project was to minimize where and how task management is done. 
To manage the product- and sprint backlogs we, initially, decided to use a spreadsheet in Google docs and Trello. The spreadsheet were used to plan the sprints, break down tasks, do estimation and assignment, whereas, Trello was used to keep track of the progress of the tasks. However, the customer wanted us to use issues actively on GitHub. We then started using GitHub task management features in addition to our already existing methods for doing so. Instead of helping us, this simply led to confusing issues and team members not knowing where to go to find issues. Different team members used different channels to update their progress, and Trello was quickly abandoned by most.

The reason we did not use GitHub to manage our project completely, even after having it suggested by the customer, was that we did not want to add tasks relating to the report to our repository on GitHub. We believed it was in the customers interest to keep them separate. However, towards the end of the project, we realized that the customer thought we managed all our tasks and backlog through GitHub, and was surprised to learn that we had another system we used as well. We all agreed that this made it difficult for the product owner to keep track of what was going on. 

Had we known about all the task management features of GitHub at the start of the project, we see, in hindsight, that we should have simply managed our project through this one channel from the start. This insight into GitHub and how it also eased communications with the customer has been a great lessons from the project.
\section{Customer and Requirements}
The team members were unfamiliar with Lean principles of development, and the biggest challenge initially was to understand this way of working. Expecting to quickly have established clear requirements to base our work on, it was difficult for the team to respond to what we, at the time, felt was an indecisive customer. We see in hindsight, that the customer's approach to the development process was one we were not familiar with. It took us several weeks to realize that we could and should be more proactive in coming up with solutions and that we were equally capable as the customer when it came to brainstorm solutions.

Had we realized this sooner, we could probably have had greater progress in a couple of sprints. Some of our placeholder items in the final solution might have been fully implemented, but this is in no way guaranteed as they were still dependent on factors outside our control. 

One example is the fitness indicator on our web site. Initially we believed we were going to collect data, do calculations and present this to the user. However, we were never able to settle with the customer which types of data should form the basis for any calculations.Later in the project, it became clear that Adapt, the research project we are a part of, where the ones making decisions about this data, but that they had not yet decided either what kind of data should be measured. Thus, even if we had been proactive is designing solutions for how to display the data, the functionality could not have been fully implemented as the product owner was not even made completely aware of the data to be collected and measured.

\section{Development Tools}

The development tools using during this project are emerging technologies that are really making its way into established businesses all across the globe. Learning to use new tools is never easy, and when only one team member had experience in using the tools, the learning curve was very steep in the beginning.

After some playing around with the tools things started working fine, but some tools proved to be harder to use than others.
\\ \\
Git may prove to be excruciating to work with if you have no strategy and plan on how to use it. Working with Git requires that you all work together and find a branching structure that works for you. This should be done as early on in the process as possible. 