\chapter{Requirements}


\section{Use Cases}

\begin{figure}[H]
\centering
\includegraphics[width = 0.75\textwidth]{Figures/UseCase1}
\caption{Use case 1}
    \label{fig:UC1}
    \end{figure}

\begin{table}[H]
% See definition at beginning of main.
\begin{tabular}{ | L{0.25\linewidth} | L{0.75\linewidth} | } 
 \hline \rowcolor{lightgray}
 Use case 1 & User registration  \\ 
 \hline
 Actors & Customer, user \\ 
 \hline
 Stakeholders & Customers, developers users \\ 
  \hline
 Primary actor & user  \\ 
 \hline
 Pre-conditions & System is up and running \\ 
 \hline
 Post-conditions & The user has registered a profile and is logged in \\ 
  \hline
 Triggers & User visits the website and clicks to register \\ 
 \hline
Flow &
\vspace{-5mm}
    \begin{enumerate}[noitemsep]
  \item User enters the webpage.
  \item User navigates mouse to register button and clicks
  \item User fills in user information and submits the data
  \item The user has now created a profile and is logged in
   \end{enumerate}\\ 
 \hline
 Main success scenarios & User now has a profile registered and can use it to log in in the future \\ 
 \hline
 Alternative paths & 3a. User fails to register\\
 \hline
\end{tabular}
\caption{Use Case 1}
\end{table}

\begin{figure}[H]
\centering
\includegraphics[width = 0.75\textwidth]{Figures/UseCase2}
\caption{Use case 2}
    \label{fig:UC2}
    \end{figure}

\begin{table}[H]
\begin{tabular}{ | L{0.25\linewidth} | L{0.75\linewidth} | } 
 \hline \rowcolor{lightgray}
 Use case 2 & User log in  \\ 
 \hline
 Actors & Customer, user \\ 
 \hline
 Stakeholders & Customer, developers, user \\ 
  \hline
 Primary actor & User  \\ 
 \hline
 Pre-conditions & User has a registered a profile \\ 
 \hline
 Post-conditions & User has successfully logged in \\ 
  \hline
 Triggers & User clicks on the login button on the website  \\ 
 \hline
Flow & 
    \vspace{-5mm}
    \begin{enumerate}[noitemsep]
  \item User enters the webpage.
  \item User navigates mouse to log in button and clicks
  \item User fills in user information and clicks login
  \item The user has now logged in
   \end{enumerate}\\ 
 \hline
 Main success scenarios & User is now logged in and can use the service as a registered user \\ 
 \hline
 Alternative paths & 3a. The login fails\\
 \hline
\end{tabular}
\caption{Use Case 2}
\end{table}

\begin{figure}[H]
\centering
\includegraphics[width = 0.75\textwidth]{Figures/UseCase3}
\caption{Use case 3}
    \label{fig:UC3}
    \end{figure}

\begin{table}[H]
\begin{tabular}{ | L{0.25\linewidth} | L{0.75\linewidth} | } 
 \hline \rowcolor{lightgray}
 Use case 3 & User invites friend to self-created event  \\ 
 \hline
 Actors & Customer, user, other users \\ 
 \hline
 Stakeholders & Customer, developers, user, other users \\ 
  \hline
 Primary actor & User  \\ 
 \hline
 Pre-conditions & User is logged in and ready to create event \\ 
 \hline
 Post-conditions & User created an event and invited friends to it \\ 
  \hline
 Triggers & A button to create event is clicked after all the necessary information is submitted  \\ 
 \hline
Flow & 
    \vspace{-5mm}
    \begin{enumerate}[noitemsep]
  \item Click create new event button
  \item The user fills in relevant information for the event and clicks create
  \item The user searches for friends in their friends list and invites friends to his newly created event
   \end{enumerate}\\ 
 \hline
 Main success scenarios & An event is successfully created and friends are invited to it \\ 
 \hline
 Alternative paths & 2a. The user fails to register the event.
3a. The user does not manage to invite any friends.\\
 \hline
\end{tabular}
\caption{Use Case 3}
\end{table}

\begin{figure}[H]
\centering
\includegraphics[width = 0.75\textwidth]{Figures/UseCase4}
\caption{Use case 4}
    \label{fig:UC4}
    \end{figure}

\begin{table}[H]
\begin{tabular}{ | L{0.25\linewidth} | L{0.75\linewidth} | } 
 \hline \rowcolor{lightgray}
 Use case 4 & User downloads the app  \\ 
 \hline
 Actors & Customer, user \\ 
 \hline
 Stakeholders & Customer, developers, user \\ 
  \hline
 Primary actor & User  \\ 
 \hline
 Pre-conditions & User has a smart phone/tablet \\ 
 \hline
 Post-conditions & User successfully downloaded the app \\ 
  \hline
 Triggers & The user navigates to the “download app” page and follows the link to their preferred app store.  \\ 
 \hline
Flow & 
   \vspace{-5mm}
    \begin{enumerate}[noitemsep]
  \item User is on the web page and clicks the download app tab
  \item The user navigates to the preferred app store from the links provided
  \item Downloads the app
   \end{enumerate}\\ 
 \hline
 Main success scenarios & The app is downloaded to the user’s device \\ 
 \hline
 Alternative paths & 2a. The app store is down or not available \\
 & 3a. The user is unable to download the app\\
 \hline
\end{tabular}
\caption{Use Case 4}
\end{table}

\begin{figure}[H]
\centering
\includegraphics[width = 0.75\textwidth]{Figures/UseCase5}
\caption{Use case 5}
    \label{fig:UC5}
    \end{figure}

\begin{table}[H]
\begin{tabular}{ | L{0.25\linewidth} | L{0.75\linewidth} | } 
 \hline \rowcolor{lightgray}
 Use case 5 & User creates event  \\ 
 \hline
 Actors & Customer, user \\ 
 \hline
 Stakeholders & Customer, developers, user \\ 
  \hline
 Primary actor & User  \\ 
 \hline
 Pre-conditions & User is logged in \\ 
 \hline
 Post-conditions & User created event \\ 
  \hline
 Triggers & The user clicks create event button  \\ 
 \hline
Flow & 
    \vspace{-5mm}
    \begin{enumerate}[noitemsep]
  \item User is on the web page and is logged in and navigates to the dashboard
  \item User clicks the create event button 
  \item User fills in information about the event and clicks submit
   \end{enumerate}\\ 
 \hline
 Main success scenarios & The user has successfully created a new event. \\ 
 \hline
 Alternative paths & 3a. User fails to create event\\
 \hline
\end{tabular}
\caption{Use Case 5}
\end{table}

\begin{figure}[H]
\centering
\includegraphics[width = 0.5\textwidth]{Figures/UseCase6}
\caption{Use case 6}
    \label{fig:UC6}
    \end{figure}

\begin{table}[H]
\begin{tabular}{ | L{0.25\linewidth} | L{0.75\linewidth} | } 
 \hline \rowcolor{lightgray}
 Use case 6 & User adds friend to network  \\ 
 \hline
 Actors & Customer, user \\ 
 \hline
 Stakeholders & Customer, developers, user \\ 
  \hline
 Primary actor & User  \\ 
 \hline
 Pre-conditions & User is logged in \\ 
 \hline
 Post-conditions & User added friend to their network \\ 
  \hline
 Triggers & The user clicks search for other users button  \\ 
 \hline
Flow & 
    \vspace{-5mm}
    \begin{enumerate}[noitemsep]
  \item User in on the web page and navigates to the search bar
  \item User clicks the search to initiate the search
  \item User finds the person it wants to add and clicks add to friends list
   \end{enumerate}\\ 
 \hline
 Main success scenarios & The user has successfully added a person to their network \\ 
 \hline
 Alternative paths & 3a. User fails to add contact to their friends list\\
 \hline
\end{tabular}
\caption{Use Case 6}
\end{table}


\section{Functional Requirements}
Based on the use cases, we settled on the following functional requirements: 
\begin{center}
\begin{table}[H]
    \centering
\begin{tabular}{ |C{0.05\linewidth}|C{0.1\linewidth}|C{0.15\linewidth}|L{0.7\linewidth}| } 
 \hline \rowcolor{lightgray}
 \multicolumn{4}{|c|}{Functional Requirements} \\
 \hline
 No. & Priority & Use case no. & Description \\
 \hline
 F1 & P1 & UC1 & Register user profile \\ 
 \hline
 F2 & P1 & UC2 & Registered user should be able to login \\ 
  \hline
 F3 & P3 & UCX & Registered user should be able to edit their profile \\
  \hline
 F4 & P1 & UC3 & Registered user can create event \\
  \hline
 F5 & P1 & UC3 & Registered user can search for events \\
  \hline
 F6 & P1 & UC3 & Registered user can invite friends to event \\
  \hline
 F7 & P1 & UCX & Registered user can join others events \\
  \hline
 F8 & P2 & UCX & Non-registered user can view public events \\
  \hline
 F9 & P2 & UCX & Registered user can view their fitness information in his/her profile page \\
  \hline
 F10 & P2 & UC5  & Registered user can manually add their activities \\
  \hline
 F11 & P1 & UC3  & Registered user can search for and view other users’ profiles \\
  \hline
 F12 & P2 & UC3  & Registered user can view friend list \\
  \hline
 F13 & P2 & UCX  & Registered user can add/delete friends from friend list \\
  \hline
 F14 & P3 & UCX  & Registered users can instant message other users \\
  \hline
 F15 & P1 & UC4  & All users should be able to download the app from the website \\
  \hline
 F16 & P3 & UCX  & All users should be able to view different exercises on the web page \\
  \hline
 F17 & P3 & UCX  & All users should be able to take a simple fitness test on the web site\\
  \hline
 F18 & P2 & UCX  & Registered users should get updates from other users in a news feed \\
  \hline
 F19 & P2 & UCX  & All users should be able to get basic information about fall risk \\
 \hline
\end{tabular}
\caption{Functional Requirements}
 \label{table:funcReq}
\end{table}
\end{center}


\section{Non-Functional Requirements}

\begin{table} [H]\centering
    \begin{tabular}{ |C{0.05\linewidth}|C{0.15\linewidth}|L{0.7\linewidth}| } 
 \hline \rowcolor{lightgray}
 \multicolumn{3}{|c|}{Non-Functional Requirements} \\
 \hline
 No. & Name & Description \\
 \hline
 NF1 & Usability & As the service is designed with the elderly in mind, simple and easily understandable interfaces is a must \\ 
 \hline
 NF2 & Security & Secure personal data like username, password, email and fitness information \\ 
  \hline
 NF3 & Accessibility & The users should be able to access the system \\
  \hline
 NF4 & Availability & The system is up and running at all times \\
  \hline
 NF5 & Maintainability & Clean and concise code that can be maintained and extended easily if needed. \\
  \hline
 NF6 & Performance & The system must be fast and responsive \\
 
 \hline
\end{tabular}
\caption{Non-Functional Requirements}
\end{table}

\section{Changes in Requirements(Change Order?)}
\subsection{Additions}
\begin{enumerate}
    \item Users of the system should be distinguished between administrators and non-administrators.
    \item Administrators must be able to create user accounts
    \item Administrators must be able to create exercises
    \item Administrators must be able to enter personal data about users
    \item Users must be able to select exercises and add them to their accounts
    \item Users must be able to add exercises to events
\end{enumerate}

\subsection{Removals}
Only one functional requirement was removed halfway through the project. This was 
\begin{enumerate}
    \item F8: \textit{Non-registered users can view public events}
    \begin{itemize}
        \item[-] Replaced by public events only being visible to registered users.
    \end{itemize}
\end{enumerate}
 
No other functional requirements were removed, but parts of the functionality in the original requirements were replaced with placeholders in the final system. This was in agreement with the customer, as the ADAPT project was not able to deliver the necessary data needed to implement the functionality. The requirements affected were: 
\begin{enumerate}
    \item  F9: \textit{Users can view their own fitness information.} The fitness information should have consisted of diagrams and graphs and a composite indicator displaying the general fitness based on several different parameter. Since it never became clear which parameters would form the basis for this fitness indicator, the final solution contains a placeholder fitness indicator, but no functionality is attached to this indicator. 
    \item F17: \textit{Take a simple fitness test.} The fitness test is also a placeholder test. The parameters necessary for creating a trustworthy test were never supplied.
    
\end{enumerate}


\section{Requirements Evaluation??}

// Should we include a section on Requirements Evalutation?

