

\chapter{Introduction}
\section{Project Motivation}
It is estimated that falls among the elderly cost Europe \euro{} 25 billion every year \cite{eupha}, and Norway spends 2.7 billion NOK each year on fall-related injuries. The individual effects of falls include \citep{fallforebygging}: 
\begin{itemize}
  \item 1800 deaths per year in Norway.
  \item 9000 hip fractures per year in Norway.
  \item long periods of hospitalization and
  \item loss of independence
\end{itemize}

It is clear that reducing the number of annual falls would greatly benefit society, and this is the problem presented to us by our customers and the basis of our project purpose.

\section{Project Purpose}
The purpose of the course IT2901 is, as stated on the course website, for students “to gain practical experience with the development of a software process for a customer, covering the whole life-cycle of the software project.”\citep{course} The specific goal of our team’s project is to create a web service with an additional app for our customer at SINTEF with the aim of preventing falls among senior citizens. 

\section{The Customer}

The customer is the research institution SINTEF and one of its researches Babak Farshchian. SINTEF, \textit{Stiftelsen for industriell og teknisk forskning ved Norges Tekniske Høgskole}, was founded in  1950, and is with its 1700 employees a “broadly based, multidisciplinary research institute with international top-level expertise in technology, medicine and the social sciences.”\citep{sintef}

Our contact person, Babak Farshchian is a researcher and research manager at SINTEF in the fields of software engineering, safety and security.  He is also an associate professor at NTNU and has previously been a customer for other students with similar projects.

\section{The Problem}

The problem we are going to solve for the customer is to reduce falls among the elderly. Injuries from falling can lead to loss of independence, long periods of hospitalization and early death and it is estimated that falls among the elderly cost Europe \euro{} 25 billion every year \cite{eupha}.  Our objective with this project is to reduce the number of falls among the elderly. To do this, the customer wants us to develop a web portal which will be informative and preventative relating to fall injuries. From the web portal the users should be able to download an app (potentially several) which will track health information and allow health personnel to review the users’ fitness levels and, consequently, their risk of falling.

Research suggests that doing regular exercise helps to reduce falls, so the main problem we need to solve is how to motivate senior citizens to starts exercising. Secondly, we need to research how to ensure continued use of the web-portal for more than a few weeks. For the web-portal to have an impact, the usage need to be sustained over a longer time period. Several aspects to retain users include making the solution social, fun, non-stigmatizing and addictive. 
\section{Stakeholders}
The two main stakeholders in this project are the customer, Babak Farshchian, and a project he is leading called Adapt. The partners in the Adapt project are St. Olav's Hospital HF, SINTEF, and Trondheim municipality. The purpose of the project is to develop and test different technologies to assess the fall risk among individual seniors citizens.\citep{adapt}

\section{Product Owner}
The customer is also the product owner of the project. As he is the both leader of the Adapt project and a researcher at SINTEF, he has close ties to several of the stakeholders and can be a clear voice in representing their views regarding our project.

\section{The Team}
The team consists of seven students in their 6th term of an undergraduate degree in Computer Science. With highly varying schedules and none of the team members knowing each other prior to the semester, some issues had to be solved to find efficient ways of collaborating. Although most of us have pursued the same degree and attended the same courses, the competencies within the group varied. Half the group has programmed extensively outside of their studies, while the rest has not. The effect of this, however, has been negligible. Here follows a list of the different group members and their competencies:

\paragraph*{Emil Sundvall Andersen:} Java, Python, HTML/CSS, JavaScript, SQL
\paragraph*{Aleksander Foosnæs:} Java, Python, HTML/CSS, JavaScript, SQL
\paragraph*{Christoffer Jahren:} Java, Python, HTML/CSS, JavaScript, SQL, Ionic
\paragraph*{Sindre Berntsen Skarås:} Java, Python, HTML/CSS, JavaScript, SQL
\paragraph*{Sarah Svedenborg:} Java, Python, HTML/CSS, JavaScript, SQL, Android
\paragraph*{Brang Nu Bok Tong:} Java, Python, HTML/CSS, JavaScript, SQL
\paragraph*{Eirik Wist:} Java, Python, HTML/CSS, JavaScript, SQL


\section{Report Structure}
Chapter one provides an introduction to the course and problem to be solved throughout the project.  

Chapter two presents the pre-study the team did at the beginning of the project to gain a better understanding of the problem to be solved. The two main areas of focus are fall prevention research and senior usability.

Chapter three proposes a first iteration of the concept for the solution based on the pre-study.

The requirements of the solution are then defined in chapter four along with use cases and changes in requirements during the project.

Chapter five evaluates already existing solutions addressing similar issues to the problem we aim to solve. These are evaluated based on a selection of our functional requirements.

Chapter six presents the desired solution reached by the team together with the customer. This solution is based on the pre-study, requirements and evaluation of existing solutions.

Chapter seven outlines the project management including choice of process model, risk analysis, tools and team organization. The chapter ends with an outline of all the sprints.

Chapter eight describes the development environment, describing all the development tools used in the project.

Chapter nine outlines the system design including architecture, system components and state diagrams.

Chapter ten describes the test strategy. Here, all the different levels and types of testing are described.

Chapter eleven is the evaluation of the solution. This addresses which requirements are met and which are not.

Chapter twelve presents the suggested future improvements to the system.

Chapter thirteen evaluates the project experience as a whole focusing on the lessons learned.

Chapter fourteen concludes the report and the project. 
