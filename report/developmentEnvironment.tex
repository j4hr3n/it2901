\chapter{Development Environment}
\section{Development Tools}
\subsection{Version Control - Git}
To manage different versions of our product code, we used the "free and open source distributed version control system "\cite{git} git. With seven team members working on the same code, the branching feature is especially important.
As many of the team members have limited experience with bigger software development projects, the fact that each team members has their own copy of the repository is a safety mechanism to prevent code corruption. Local repositories and branching allows team members to experiment with and test code, while not having to worry about corrupting the project. 
However, code corruption can always occur, and in the event of this, the use of git would make it unproblematic to restore the code to a working state. 
\subsection{Prototyping - Balsamiq Mock-ups}
Balsamiq mock-ups is a  graphical user interface mockup builder application. It allows the designer to arrange pre-built widgets using a drag-and-drop WYSIWYG editor to quickly and easily create mockups and wireframes. A large part of this project concerns itself with coming up with concepts of how to reach the senior population. In order to test different ideas and give the customer a feel of what we are thinking, Balsamiq let’s us more visually display our ideas. In addition, the customer firmly believes in iterative development, and would like us to quickly come up with concrete ideas so that concrete feedback can be given. Balsamiq lets us easily change our wireframes to quickly be able to display new ideas. 
\subsection{Text Editors - Atom and Sublime Text}
These are the text editors that we use to edit our codes. Some team members prefer Atom and the others Sublime Text. We chose these two to use in our project because they are de-facto standard software programs that many programmers use to write and edit code. Syntax highlighting and code completion features are very useful for reducing the amount of work that needs to be done. This saves us a lot of time to focus on other areas of the project such as report, design, testing and deployment. 
\subsection{Frameworks for web solution}
\subsubsection{Angular}
Angular is a web software framework that is supposed to enhanced HTML5 web application and Single Page Application(SPA). The framework is actively developed and supported by Google. There are two versions of Angular, 1 and 2. For our project, we decided to use Angular 1 since it is widely used and community support for it is very good in the Angular ecosystem. 
\subsubsection{Meteor}
Meteor is a full stack Javascript application framework developed by Meteor Development Group. It is full stack because it uses Javascript on both Frontend and Backend sides. Node.js is used on the backend to provide server and file management. For the database, Meteor only support NoSQL variant of MongoDB. Since Javascript is a very well known language to every team member in our group, we choose Meteor to rapidly develop the prototype and the product itself. This saves us tremendous amount of time because we can use Javascript on both frontend and backend. 
\subsection{Framework for mobile app solution}
The Meteor framework used for the web solution is also used for the mobile application, but in addition, Cordova and the Ionic framework  are used to create the moblie app.
\subsubsection{Cordova}
Cordova is a mobile development framework created to give developers access to native APIs using web technology. This means that web developers can use standard web technology such as HTML, CSS and Javascript to development mobile applications. Since Cordova provides native APIs access for both iOS and Android, developers can easily use these native APIs by using Javascript to develop mobile apps for both iOS and Android. Normally one needs to learn both Swift or Objective-C to develop iOS app and Java for Android. By using Cordova, we can write our code using web technologies once and run it on almost every mobile operating system such as Android, iOS and Windows. 
\subsubsection{Ionic}
Ionic is a mobile application development framework used to create mobile apps. Underneath its architecture, Ionic uses Cordova to access native APIs such as camera, geo-location and notification system. Besides that Ionic lets us use Angular in its environment to rapidly development mobile prototypes and application. We use Meteor and AngularJS for our web application and Ionic, Cordova and Angular for iOS and Android mobile application.