\chapter{Pre-Study}
\section{Fall Prevention Research}

In order to gain a deeper understanding of the problem, three areas in particular became the object of our research on the matter. These include statistics about falls, causes for falls, and existing measures of fall prevention. The European Network for Safety among Elderly (EUNESE) provided much of the needed information. Below follows a summary of our findings. 

\subsection*{Statistics about falls}
\begin{itemize}
  \item Falls are the leading cause of injury among people aged 65 and older \cite{facts}
  \item Older adults who fall once are more likely to fall again within a year\cite{conference}
  \item 50 percent of falls among elderly occur at home\cite{cerepri}
  \item The major source of hospital costs are fractures, mainly of the hip \cite{Polinder}
\end{itemize}

\subsection*{Causes of falls}
\begin{itemize}
    \item[] \textbf{Individual}\cite{facts}
        \begin{itemize}
          \item[•] Age 
          \item[•] Living alone
          \item[•] Multiple medications
          \item[•] Impaired mobility
          \item[•] Fear of falling
          \item[•] visual impairments
        \end{itemize}
    \item[] \textbf{Environmental}\cite{facts}
         \begin{itemize}
          \item[•] Environmental hazards (slippery floors, poor lighting etc)
          \item[•] Inappropriate footwear of clothing
          \item[•] Inappropriate walking aids
        \end{itemize}
    \item[] \textbf{Exposure to risk}\cite{facts}
         \begin{itemize}
          \item[•] Some studies suggest that the most inactive and the most active people are at the highest risk of falls
          \item[•]Specific activities seem to increase the risk of falls, either by increasing exposure to risky environmental conditions (slippery or uneven floors, cluttered areas, degraded pavements), acute fatigue, or unsafe practice in exercise sessions
        \end{itemize}
\end{itemize}

\subsection*{Fall prevention}
Also based on EUNESE's \cite{facts} research:
\begin{itemize}
  \item Physical activity and balance training promotion
  \item Medical review
  \item Dietary supplements
  \item Vision assessment and modification
  \item Feet and footwear review
  \item Home modification
\end{itemize}

\section{Senior Usability Research}
The focus on usability and user experience is always important in web solutions, but what creates great usability will vary form project to project depending on \textit{who} are the users of the system, \textit{what} they will do and in which \textit{context} they will do it. As this project concerns itself with reducing falls among the elderly in society, the demographic of our solution will be seniors. In addition to researching general rules for good usability, we researched specific usability guidelines for seniors and found that Dana E. Chisnell, Janice C. Redish and Amy Lee \cite{heuristics} have researched the area thoroughly and come up with 20 heuristics important for web sites aimed at seniors. Based on these we came up with a set of several aspects to pay attention to: 
\begin{itemize}
  \item Contrast is key regarding colour use
  \item Sans-serif fonts are more easily processed
  \item Large font
  \item Conventional interaction elements: links and buttons should behave in an expected way and be consistent throughout the whole web site
  \item Make obvious what is clickable and what is not
  \item Make clickable items easy to target and hit
  \item Minimize vertical scrolling
  \item Eliminate horizontal scrolling
  \item Ensure that the back button behaves predictably
  \item Let the user stay in control
  \item Provide clear feedback on actions
  \item Provide feedback in other modes in addition to visual
  \item Make structure of website as visible as possible
  \item Clearly label content categories
  \item Shallowest possible information hierarchy
  \item Include site map and link to it from every page
  \item Use adequate whitespace
  \item Minimize jargon and technical terms
  
\end{itemize}
