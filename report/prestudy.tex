\chapter{Pre-Study}
\section{Fall Prevention Research}
\section{Senior Usability Research}
The focus on usability and user experience is always important in web solutions, but what creates great usability will vary form project to project depending on \textit{who} are the users of the system, \textit{what} they will do and in which \textit{context} they will do it. As this project concerns itself with reducing falls among the elderly in society, the demographic of our solution will be seniors. In addition to researching general rules for good usability, we researched specific usability guidelines for seniors and found that Dana E. Chisnell, Janice C. Redish and Amy Lee \cite{heuristics} have researched the area thoroughly and come up with 20 heurisistics important for web sites aimed at seniors. Based on these we came up with a set of several aspects to pay attention to: 
\begin{itemize}
  \item Contrast is key regarding colour use
  \item Sans-serif fonts are more easily prosessed
  \item Large font
  \item Conventional interaction elements: links and buttons should behave in an expected way and be consistent throughout the whole web site
  \item Make obvious what is clickable and what is not
  \item Make clickable items easy to target and hit
  \item Minimize vertical scrolling
  \item Eliminate horizontal scrolling
  \item Ensure that the back button behaves predictably
  \item Let the user stay in control
  \item Provide clear feedback on actions
  \item Provide feedback in other modes in addition to visual
  \item Make structure of website as visible as possible
  \item Clearly label content categories
  \item Shallowest possible information hierarchy
  \item Include site map and link to it from every page
  \item Use adequate whitespace
  \item Minimize jargon and technical terms
  
\end{itemize}
